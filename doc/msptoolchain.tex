\section{MSP430-Toolchain (unter Linux)}
\subsection{Installation}
Die Installation der MSP-430 Toolchain ist unter Ubuntu bereits in den 
Paketquellen vorhanden (ab Oneric Ocelot aka 11.10)

\begin{itemize}
    \item binutils-msp430 
    \item gcc-msp430 
    \item gdb-msp430 
    \item msp430-libc 
    \item msp430mcu 
    \item mspdebug
\end{itemize}

\noindent
Da hier Pakete verwendet werden, die Manipulationen am GDB vornehmen gilt es
folgende Eingabe zu machen, falls man GDB bereits intalliert hat \\

\verb?apt-get -o Dpkg::Options::="--force-overwrite" install gdb-msp430?

\subsection{Kompilation}
Die Kompilation erfolgt durch \emph{msp430-gcc} und als Option wird der MCU Typ
angegeben (beim MSP430 LaunchPad ist dies \emph{-mmcu=msp430g2553}). Die weiteren
Angaben sind analog zum GCC, d.h. \emph{-o Output-File ./Input-File.c}.
Hier nochmal ein Beispiel \\

\verb?msp430-gcc -mmcu=msp430g2553 -o myprog ./myprog.c?

\subsection{Auf Flash schreiben}
Das Programm wird mittels des MSP-Debugger auf das Flash übertragen. Hierzu
ruft man \emph{mspdebug} und gibt den Treiber an (im Falle des LaunchPad ist
dies der \emph{rf2500}).\\

\verb?mspdebug rf2500?\\

\noindent
Dies eröffnet die Konsole und stellt so viele Kommandos bereit. Um ein 
Übertragen auf das LaunchPad zu ermöglichen, muss der Konsole zuerst das
betreffende Binary angegeben werden.\\

\verb?(mspdebug) prog myprog?\\

\noindent
Nachdem der Debugger das Binary kennt, schreibt er es sofort auf den Flash.
Damit das Programm ausgeführt wird, muss dem Debugger noch das Kommando 
\emph{run} übergeben werden.\\

\verb?(mspdebug) run?

\subsection{Debug \& Simulation}
\emph{mspdebug} bietet wie es der Name schon verrät Debug-Möglichkeiten.
Hierzu wird \emph{mspdebug} mit der Option \emph{sim} ausgeführt und
dann in dessen Konsole \emph{gdb} eingegeben.

\subsection{Quellen \& Testing}
Die hier angegebenen Informationen sind ein Auszug aus dem Artikel
\href{http://wiki.ubuntuusers.de/MSP430-Toolchain}{MSP-430 Toolchain} von 
Ubuntuusers. 

Die hier oben genannten Anleitungen sind mit folgender Konfiguration
erfolgreich gestestet worden an einem MSP-EXP430G2 LaunchPad Rev 1.5\footnote{
    Für Luxeria LuXeria-Member stehen mehrere MSP430-LaunchPads und eine
    EZ430-Chronos frei zur Verfügung im LuXlab.}:

\begin{table}[h!]
\centering
\begin{tabular}{ l l }
Hardware        & Lenovo ThinkPad 430 \\
Kernel-Name     & Linux \\
Kernel-Release  & 3.5.0-17-generic \\
Kernel-Version  & \#28-Ubuntu SMP Tue Oct 9 19:32:08 UTC 2012 \\
Machine         & i686 \\
Operatin System & GNU/Linux\\
\end{tabular}
\end{table}
