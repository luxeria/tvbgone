% coding:utf-8

%TV-B-Gone
%Copyright (C) 2013, Daniel Winz, Ervin Mazlagic

%This program is free software; you can redistribute it and/or
%modify it under the terms of the GNU General Public License
%as published by the Free Software Foundation; either version 2
%of the License, or (at your option) any later version.

%This program is distributed in the hope that it will be useful,
%but WITHOUT ANY WARRANTY; without even the implied warranty of
%MERCHANTABILITY or FITNESS FOR A PARTICULAR PURPOSE.  See the
%GNU General Public License for more details.
%----------------------------------------

\section{Hardware Konventionen}
Um in Hardware Projekten eine gewisse Kontinuität zu gewährleisten sollten 
gewissen grundsäliche Konventionen definiert werden. 

\subsection{Schema}

\subsubsection{Bauteilbezeichnung}
Damit in allen Projekten das gleiche Bauteil jeweils die gleiche Referenz 
besitzt, wird dies hier festgelegt. 

Die nachfolgende Tabelle ist erst ein Vorschlag für eine mögliche Konvention. 
\begin{table}[h!]
  \begin{tabular}{ll}
  \textbf{Bauteil} & \textbf{Referenz}\\
  Widerstand & R \\
  Kondensator & C \\
  Spule & L \\
  
  \end{tabular}
\end{table}

\subsection{Layout}