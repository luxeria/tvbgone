% coding:utf-8

%TV-B-Gone
%Copyright (C) 2013, Daniel Winz, Ervin Mazlagic

%This program is free software; you can redistribute it and/or
%modify it under the terms of the GNU General Public License
%as published by the Free Software Foundation; either version 2
%of the License, or (at your option) any later version.

%This program is distributed in the hope that it will be useful,
%but WITHOUT ANY WARRANTY; without even the implied warranty of
%MERCHANTABILITY or FITNESS FOR A PARTICULAR PURPOSE.  See the
%GNU General Public License for more details.
%----------------------------------------

\section{Hardware Konventionen}
Um in Hardware Projekten eine gewisse Kontinuität zu gewährleisten sollten 
gewissen grundsätzliche Konventionen definiert werden. 

\subsection{Schema}

\subsubsection{Bauteilbezeichnung}
Damit in allen Projekten das gleiche Bauteil jeweils die gleiche Referenz 
besitzt, wird dies hier festgelegt. 

Die nachfolgende Tabelle ist erst ein Vorschlag für eine mögliche Konvention. 

Die Angaben aus den Normen stammen von 
\url{http://www.elektronik-kompendium.de/sites/slt/1204031.htm}. 
\begin{table}[h!]
  \begin{tabular}{llll}
  \rowcolor{white}  \textbf{Bauteil} 	    & DIN 40719-2   & DIN EN 81346-2    & \textbf{Vorschlag}\\
  \rowcolor{lgray}  Widerstand              & R             & R                 & R \\
  \rowcolor{white}  Kondensator             & C             & C                 & C \\
  \rowcolor{lgray}  Spule                   & L             & R                 & L \\
  \rowcolor{white}  Stecker                 &               &                   & J oder Con \\
  \rowcolor{lgray}  Sicherung               & F             & F                 & F \\
  \rowcolor{white}  Transistor              & V             & K                 & T \\
  \rowcolor{lgray}  Diode                   & V             & R                 & D \\
  \rowcolor{white}  Quarz                   &               &                   & Q oder X \\
  \rowcolor{lgray}  Transformator           & T             & T                 & T oder L \\
  \rowcolor{white}  IC                      &               &                   & U, V oder IC \\
  \rowcolor{lgray}  Potentiometer           & R             & R                 & R oder Pot \\
  \rowcolor{white}  Jumper                  &               &                   & J \\
  \rowcolor{lgray}  Display                 &               &                   & Disp \\
  \rowcolor{white}  Motor                   & M             & M                 & M \\
  \rowcolor{lgray}  Taste / Schalter        & S             & S                 & S \\
  \rowcolor{white}  Relais                  & K             & K                 & K \\
  \rowcolor{lgray}  Kühlkörper              &               &                   & Hs \\
  \rowcolor{white}  LED                     & V             & P                 & D, V oder LED \\
  \rowcolor{lgray}  Batterie / Akku         & G             & G                 & Bat \\
  \rowcolor{white}  Elektronenröhre         &               &                   & ? \\
  \rowcolor{lgray}  Lautsprecher / Buzzer   &               &                   & Spk \\
  \rowcolor{white}  Mikrofon                &               &                   & Mic \\
  \rowcolor{lgray}  Abschirmblech           &               &                   & Cave \\
  \rowcolor{white}  Thyristor               & T             & Q                 & T \\
  \rowcolor{lgray}  Triac                   &               &                   & T \\
  \rowcolor{white}  IGBT                    &               &                   & T \\
  \rowcolor{lgray}  FET                     &               &                   & T \\
  \rowcolor{white}  Markierungen            &               &                   & Mark \\
  \rowcolor{lgray}  Mechanik                &               &                   & Mech
  \end{tabular}
\end{table}

\subsection{Layout}