% coding:utf-8

%TV-B-Gone
%Copyright (C) 2013, Daniel Winz, Ervin Mazlagic

%This program is free software; you can redistribute it and/or
%modify it under the terms of the GNU General Public License
%as published by the Free Software Foundation; either version 2
%of the License, or (at your option) any later version.

%This program is distributed in the hope that it will be useful,
%but WITHOUT ANY WARRANTY; without even the implied warranty of
%MERCHANTABILITY or FITNESS FOR A PARTICULAR PURPOSE.  See the
%GNU General Public License for more details.
%----------------------------------------

\section{MSP430-Toolchain (unter Linux)}
\subsection{Installation}
Die Installation der MSP-430 Toolchain ist unter Ubuntu bereits in den 
Paketquellen vorhanden (ab Oneric Ocelot aka 11.10)

\begin{itemize}
    \item binutils-msp430 
    \item gcc-msp430 
    \item gdb-msp430 
    \item msp430-libc 
    \item msp430mcu 
    \item mspdebug
\end{itemize}

\noindent
Da hier Pakete verwendet werden, die Manipulationen am GDB vornehmen gilt es
folgende Eingabe zu machen, falls man GDB bereits intalliert hat \\

\verb?apt-get -o Dpkg::Options::="--force-overwrite" install gdb-msp430?

\subsection{Kompilation}
Die Kompilation erfolgt durch \emph{msp430-gcc} und als Option wird der MCU Typ
angegeben (beim MSP430 LaunchPad ist dies \emph{-mmcu=msp430g2553}). Die weiteren
Angaben sind analog zum GCC, d.h. \emph{-o Output-File ./Input-File.c}.
Hier nochmal ein Beispiel \\

\verb?msp430-gcc -mmcu=msp430g2553 -o myprog ./myprog.c?

\subsection{Auf Flash schreiben}
Das Programm wird mittels des MSP-Debugger auf das Flash übertragen. Hierzu
ruft man \emph{mspdebug} und gibt den Treiber an (im Falle des LaunchPad ist
dies der \emph{rf2500}).\\

\verb?mspdebug rf2500?\\

\noindent
Dies eröffnet die Konsole und stellt so viele Kommandos bereit. Um ein 
Übertragen auf das LaunchPad zu ermöglichen, muss der Konsole zuerst das
betreffende Binary angegeben werden.\\

\verb?(mspdebug) prog myprog?\\

\noindent
Nachdem der Debugger das Binary kennt, schreibt er es sofort auf den Flash.
Damit das Programm ausgeführt wird, muss dem Debugger noch das Kommando 
\emph{run} übergeben werden.\\

\verb?(mspdebug) run?

\subsection{Debug \& Simulation}
\emph{mspdebug} bietet wie es der Name schon verrät Debug-Möglichkeiten.
Hierzu wird \emph{mspdebug} mit der Option \emph{sim} ausgeführt und
dann in dessen Konsole \emph{gdb} eingegeben.

\subsection{Quellen \& Testing}
Die hier angegebenen Informationen sind ein Auszug aus dem Artikel
\href{http://wiki.ubuntuusers.de/MSP430-Toolchain}{MSP-430 Toolchain} von 
Ubuntuusers. 

Die hier oben genannten Anleitungen sind mit folgender Konfiguration
erfolgreich gestestet worden an einem MSP-EXP430G2 LaunchPad Rev 1.5\footnote{
    Für Luxeria LuXeria-Member stehen mehrere MSP430-LaunchPads und eine
    EZ430-Chronos frei zur Verfügung im LuXlab.}:

\begin{table}[h!]
\centering
\begin{tabular}{ l l }
Hardware        & Lenovo ThinkPad 430 \\
Kernel-Name     & Linux \\
Kernel-Release  & 3.5.0-17-generic \\
Kernel-Version  & \#28-Ubuntu SMP Tue Oct 9 19:32:08 UTC 2012 \\
Machine         & i686 \\
Operatin System & GNU/Linux\\
\end{tabular}
\end{table}

\section{MSP430-Toolchain (unter Windows)}
\subsection{Installation}
Diese Anleitung befasst sich nur mit mspgcc. Entwicklungsumgebungen wie Eclipse 
werden nicht behandelt. 
\subsubsection{Download}
Die Installation ist unter Windows etwas umständlicher als unter Linux. 
Zunächst muss der mspgcc \footnote{\url{http://sourceforge.net/projects/mspgcc/
files/Windows/mingw32/}} und der msp430-gdbproxy \footnote{\url{http://
sourceforge.net/projects/mspgcc/files/Outdated/msp430-gdbproxy/2006-11-14/}} 
heruntergeladen werden. 

Aktueller als msp430-gdbproxy ist die Installation von MSPdebug. Dies wird an 
dieser Stelle zu einem späteren Zeitpunkt erläutert. 

Ausserdem müssen noch die beiden DLLs MSP430.dll und HIL.dll heruntergeladen 
werden. Diese findet man unter \footnote{\url{http://software-dl.ti.com/msp430/
msp430_public_sw/mcu/msp430/DLLv2/latest/index_FDS.html}}. Um diese jedoch 
herunterladen zu können muss man bei Texas Instruments registriert sein. Eine 
andere Möglichkeit um an die fehlenden DLLs zu gelangen ist die Installation 
der Kickstartversion von IAR Embedded Workbench von \footnote{\url{http://
processors.wiki.ti.com/index.php/IAR_Embedded_Workbench_for_TI_MSP430}}. Dazu 
ist keine Anmeldung erforderlich. 

Zudem wird der Treiber aus slac254 von \footnote{\url{?}} benötigt. 

\subsubsection{Installation Treiber}
Als Erstes muss der Treiber für das Launchpad installiert werden. Dazu muss 
slac524 entpackt werden. Anschliessend muss LaunchPad\_Driver.exe ausgeführt 
werden. Unter Windows 7 muss LaunchPad\_Driver.exe als Administrator ausgeführt 
werden. 

Wenn nun das Launchpad angeschlossen wird, muss es im Gerätemanager als 
serielle Schnittstelle angezeigt werden. 


% Die Installation der MSP-430 Toolchain ist unter Ubuntu bereits in den 
% Paketquellen vorhanden (ab Oneric Ocelot aka 11.10)

% \begin{itemize}
%     \item binutils-msp430 
%     \item gcc-msp430 
%     \item gdb-msp430 
%     \item msp430-libc 
%     \item msp430mcu 
%     \item mspdebug
% \end{itemize}

% \noindent
% Da hier Pakete verwendet werden, die Manipulationen am GDB vornehmen gilt es
% folgende Eingabe zu machen, falls man GDB bereits intalliert hat \\

% \verb?apt-get -o Dpkg::Options::="--force-overwrite" install gdb-msp430?

% \subsection{Kompilation}
% Die Kompilation erfolgt durch \emph{msp430-gcc} und als Option wird der MCU Typ
% angegeben (beim MSP430 LaunchPad ist dies \emph{-mmcu=msp430g2553}). Die weiteren
% Angaben sind analog zum GCC, d.h. \emph{-o Output-File ./Input-File.c}.
% Hier nochmal ein Beispiel \\

% \verb?msp430-gcc -mmcu=msp430g2553 -o myprog ./myprog.c?

% \subsection{Auf Flash schreiben}
% Das Programm wird mittels des MSP-Debugger auf das Flash übertragen. Hierzu
% ruft man \emph{mspdebug} und gibt den Treiber an (im Falle des LaunchPad ist
% dies der \emph{rf2500}).\\

% \verb?mspdebug rf2500?\\

% \noindent
% Dies eröffnet die Konsole und stellt so viele Kommandos bereit. Um ein 
% Übertragen auf das LaunchPad zu ermöglichen, muss der Konsole zuerst das
% betreffende Binary angegeben werden.\\

% \verb?(mspdebug) prog myprog?\\

% \noindent
% Nachdem der Debugger das Binary kennt, schreibt er es sofort auf den Flash.
% Damit das Programm ausgeführt wird, muss dem Debugger noch das Kommando 
% \emph{run} übergeben werden.\\

% \verb?(mspdebug) run?

% \subsection{Debug \& Simulation}
% \emph{mspdebug} bietet wie es der Name schon verrät Debug-Möglichkeiten.
% Hierzu wird \emph{mspdebug} mit der Option \emph{sim} ausgeführt und
% dann in dessen Konsole \emph{gdb} eingegeben.

\subsection{Quellen \& Testing}
Die hier angegebenen Informationen sind ein Auszug aus dem Artikel
\href{http://springuin.nl/articles/launchpadwindows}{TI LaunchPad on Windows} von 
Springuin. 

% Die hier oben genannten Anleitungen sind mit folgender Konfiguration
% erfolgreich gestestet worden an einem MSP-EXP430G2 LaunchPad Rev 1.5\footnote{
%     Für Luxeria LuXeria-Member stehen mehrere MSP430-LaunchPads und eine
%     EZ430-Chronos frei zur Verfügung im LuXlab.}:

\begin{table}[h!]
\centering
\begin{tabular}{ l l }
Hardware        & HP ProBook 4740s \\
Betriebssystem  & Windows 7 Professional 64 Bit \\
                & Service Pack 1 \\
% Kernel-Release  & 3.5.0-17-generic \\
% Kernel-Version  & \#28-Ubuntu SMP Tue Oct 9 19:32:08 UTC 2012 \\
% Machine         & i686 \\
% Operatin System & GNU/Linux\\
\end{tabular}
\end{table}

