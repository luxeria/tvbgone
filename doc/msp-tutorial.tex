\section{Grundlagen der Programmierung}
Um einen Mikrocontroller wie den MSP430G2553 zu programmieren, bedarf es
einiger Vorkehrungen. Diese werden oft unter dem Begriff ``Init'' 
verschleiert. Hierbei ist oft eine sogenannte Initialisierung der 
Hardware gemeint. Dies betrifft im einfachsten Fall nur den Mikrocontroller.
In dieser Initialisierung werden vielerlei Dinge definiert.

\begin{comment}
	Im folgenden wird eine solche Initialisierung vorgestellt samt C- 
	und Header-File.

	\subsection{Hardwareinitialisierung}
	\lstinputlisting{../software/tutorial/button/hardware.c}
	\lstinputlisting{../software/tutorial/button/hardware.h}
\end{comment}

\subsection{Beispielprogramme}

\subsubsection{Interrupt auf GPIO}
In diesem Beispiel wird gezeigt, wie man einen Interrupt durch einen
GPIO\footnote{\emph{General Purpose Input/Output}} anlegt.

\lstinputlisting{../software/tutorial/button/button.c}
