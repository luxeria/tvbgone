\section{Grundlagen der Programmierung}
Um einen Mikrocontroller wie den MSP430G2553 zu programmieren, bedarf es
einiger Vorkehrungen. Diese werden oft unter dem Begriff ``Init'' 
verschleiert. Hierbei ist oft eine sogenannte Initialisierung der 
Hardware gemeint. Dies betrifft im einfachsten Fall nur den Mikrocontroller.
In dieser Initialisierung werden vielerlei Dinge definiert und gerade Personen
die mit der Architektur nicht vertraut sind, scheitern oft an genau dieser
ersten aller Hürden der uC-Programmierung.

Ein gutes Beispiel einer solchen Hardware-Initialisierung ist 
auf \url{http://github.com/daniw/msp430helloworld} einzusehen.

\begin{comment}
	Im folgenden wird eine solche Initialisierung vorgestellt samt C- 
	und Header-File.

	\subsection{Hardwareinitialisierung}
	\lstinputlisting{../software/tutorial/button/hardware.c}
	\lstinputlisting{../software/tutorial/button/hardware.h}
\end{comment}

\subsection{Beispielprogramme}

\subsubsection{Interrupt auf GPIO}
In diesem Beispiel wird gezeigt wie man einen Interrupt durch einen
GPIO\footnote{\emph{General Purpose Input/Output}} anlegt.

\lstinputlisting[firstline=8, lastline=39]{../software/tutorial/button/button.c}

\subsubsection{Interrupt auf Timer}
In diesem Beispiel wird gezeiget wie man einen Interrupt durch einen internen
Timer anlegt.

\lstinputlisting{../software/tutorial/timer/timer.c}
